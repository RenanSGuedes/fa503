\newpar O planejamento agrícola diz respeito ao conjunto de medidas tomadas antes do estabelecimento da cultura numa região, baseando-se nas características do clima e sua variabilidade interanual compondo o conjunto de informações agrometeorológicas.  O planejamento se diferencia da tomada de decisão pois a última é resultado da primeira. Com base nas informações coletadas são realizadas ações de modo a contemplar o planejamento feito de antemão. Ou seja, enquanto o primeiro é realizado antes do ciclo das culturas o tomada de decisão é realizada durante este ciclo. 