\newpar O planejamento agrícola depende das condições naturais apresentadas pelo ambiente. A depender da cultura a ser semeada é feito um estudo de modo a contemplar diferentes níveis de impacto que a variação climática pode causar numa região. O ser humano ao não poder alterar o clima de uma região (macroclima), deve fazer um estudo de forma prévia visando atender os requisitos do cultivar. Dessa forma, é possível analisar a interação entre o clima e as plantas e como isso afeta a estrutura do solo ou ação de insetos e plantas daninhas no ambiente, por exemplo. Feito isso, as tomadas de decisão são resultado do planejamento, tendo em vista que ao fazer um apanhado das condições e estratégias disponíveis podem ser realizadas ações benéficas ao produtor e empresas associadas.