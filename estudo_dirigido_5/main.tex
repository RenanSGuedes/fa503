\documentclass[a4paper, 12pt]{article}

\usepackage{config}

\begin{document}
	\begin{itemize}
		\item\textbf{Nome:} Renan da Silva Guedes
		\item\textbf{RA:} 223979
	\end{itemize}

	\begin{center}
		\begin{large}
			\uppercase{\textbf{Estudo Dirigido 5}}
		\end{large}
	\end{center}

	\begin{enumerate}
		\item\textbf{Quais os diferentes tipos de geada e as condições em que cada uma delas ocorre?}
		
		Os tipos de geadas são definidos quanto à sua gênese (origem) ou pelos efeitos visuais (aspectos da planta) que elas produzem. Elas ocorrem em função de dois fenômenos meteorológicos: \textit{advecção de ar frio}, e \textit{perda de radiação terrestre}. O primeiro tipo é gerado pela ocorrência de ventos fortes podem reduzir as temperaturas na superfície foliar, ou até mesmo ocasionar a morte da planta devido ao estresse provocado pelo balanço excessivo. O segundo tipo ocorre quando a superfície terrestre perde energia em condições de baixa umidade do ar, dessa forma é ocasionado um gradiente de temperaturas diretamente proporcional à altura em relação ao solo, gerando a chamada inversão térmica.
		
		\item\textbf{Cite os fatores de formação das geadas no Brasil. Combinando-se as três escalas, qual seria a condição, ao seu ver, mais propensa à ocorrência de geadas no Brasil?}
		
		Dentre os fatores pode-se citar: Latitude, altitude, continentalidade/maritimidade e massa de ar polar. 
		
		O Brasil, por possuir dimensões continentais e ser extenso tanto latitude quanto longitudinalmente, contempla a condição de atingir altas latitudes, principalmente próximo à região Sul do país. Além disso, com a elevação da altitude nessas regiões o ambiente torna-se ainda mais propenso para a ocorrência de tal fenômeno, tendo em vista a menor temperatura e condições de umidade atmosférica ideais.
		
		\item\textbf{Qual o efeito da geada nos vegetais?}
		
		A geada normalmente ocasiona a morte dos tecidos vegetais, podendo afetar folhas, caule frutos e ramos, pro exemplo. Devido à baixa temperatura do ar pode ocorrer o congelamento dos tecidos, havendo ou não gelo em sua superfície.
		
		\item\textbf{Cite e explique uma medida para minimizar os efeitos da Geada nos vegetais.}
		
		Por meio da utilização de séries históricas pode ser feito o planejamento das épocas de plantio/semeadura de um cultivar. Dessa forma, baseando-se no modelo probabilístico, é buscado dirimir a ocorrência de temperaturas mínimas absolutas e das geadas podendo assim preservar a cultura até a fase da colheita e evitar as perdas pelas adversidades.
		
		\item\textbf{O que é Zoneamento Agroclimático?}
		
		O \textit{zoneamento agroclimático} estuda a aptidão climática de diferentes regiões visando o cultivo de espécies de interesse agrícola a partir dos potenciais e limitações apresentados pela região.
		
		\item\textbf{Qual a importância do Zoneamento Agroclimático?}
		
		Segundo o pesquisador Ary Fortes, da Embrapa Informática Agropecuária (Campinas, SP), o zoneamento agroclimático 
		
		\textit{``minimiza os riscos de perdas por adversidades climáticas incontroláveis, garantindo a capacidade de investimento do agricultor, além de ser um importante indutor de adoção de tecnologia''.}
		
		disse também que o zoneamento contribui para a redução de gastos públicos, já que a oferta de crédito pelos bancos ao produtor é garantida para o plantio em épocas corretas. (EMBRAPA, 2017)
		
		\item\textbf{Qual a diferença entre variabilidade climática e mudança climática?}
		
		Define-se variabilidade climática como uma variação das condições climáticas em torno da média climatológica. Com base nessa variação podem ser estabelecidos outros parâmetros como anomalias e mudanças climáticas. Já a mudança climática é dada pelas mudanças ocorrentes no tempo ocasionando tendências que se distanciam das médias históricas a partir da ocorrência da variabilidade climática em períodos adjacentes.
		(ANGELOCCI, 2007)
		
		\item\textbf{O que é efeito estufa?}
		
		O efeito estufa é o fenômeno decorrente da retenção de energia irradiada da superfície terrestre pelos gases de efeito estufa visando concentrar parte do calor na camada habitável da Terra. Dessa forma, é possibilitado o conforto térmico e a manutenção da vida como a conhecemos. 
		
		\item\textbf{Explique Aquecimento Global.}
		
		O aquecimento global é o fenômeno anômalo decorrente da alta concentração de gases do efeito estufa oriundo de práticas edáficas imprudentes, ocasionando o desequilíbrio energético e desregulando as temperaturas em diferentes regiões do planeta.
		
		\item\textbf{Explique Mudança Climática.}
		
		Mudança climática é o evento ocorrente em escala regional ou global ocasionando mudanças climáticas nas médias históricas observadas nos locais.
		
		\item\textbf{Como as possíveis mudanças climáticas podem interferir nas áreas agrícolas do Brasil?}
		
		Elas podem interferir nas condições de adaptação apresentadas pelo cultivar até o instante antecessor à mudança. Dessa maneira, pode ser exigido uma nova análise do comportamento do cultivar diante do novo panorama, já que as espécies vegetais podem responder de diferentes formas, podendo acarretar perdas consideráveis na produção local.
		
		\item\textbf{O estudo da Embrapa e do CEPAGRI da UNICAMP analisou o que poderia acontecer com diferentes culturas agrícolas em cenários futuros de mudanças climáticas. Escolha uma cultura de seu interesse e explique o que poderá acontecer na área de plantio dessa cultura em cenários futuros de mudanças climáticas.}
		
		Com a elevação das temperaturas o impacto pode ocasionar grandes perdas principalmente no setor de grãos do país, podendo saltar de R\$ 7,4 bilhões já em 2020  para R\$ 14 bilhões em 2070, podendo alterar profundamento o panorama agrícola e a aptidão apresentada pelas diferentes regiões.
		A cultura da soja, aliás, é que a deve ser mais afetada pela mudança do clima. O trabalho prevê uma diminuição de até 41\% na área de baixo risco ao plantio do grão em todo o país em 2070, no pior cenário, gerando prejuízos de R\$ 7,6 bilhões. Isso equivalerá à metade das perdas projetadas para a agricultura brasileira, daqui a seis décadas, como resultado do aquecimento global.
		
	\end{enumerate}

	\section{Referências}
	
	\noindent EMBRAPA. \textbf{Pesquisador mostra a importância do zoneamento agrícola}. [S. l.], 8 set. 2017. Disponível em: https://www.embrapa.br/busca-de-noticias/-/notic\newline ia/26571354/pesquisador-mostra-a-importancia-do-zoneamento-agricola\#:\~: text=\newline Como\%20ferramenta\%20de\%20gest\%C3\%A3o\%2C\%20segundo,indutor\% 20de\newline\%20ado\%C3\%A7\%C3\%A3o\%20de\%20tecnologia\%E2\%80\%9D. \newline Acesso em: 10 jul. 2020.\\
	
	\noindent ANGELOCCI, Luiz; SENTELHAS, Paulo. Introdução. In: \textbf{Variabilidade, Anomalia e Mudança Climática}: Material didático da disciplina LCE306 -Meteorologia Agrícola - Turmas 1,4,5 e 6  Departamento. de Ciências Exatas- setor de Agrometeoorlogia - ESAL/USP - 2007. [S. l.: s. n.], 2007. p. 1-2. Disponível em: https://files.cercomp.ufg.br/weby/up/68/o/variabilidade\_\_anomalia\_e\_mudan\_\_as\_clim\_\_\-ticas.pdf. Acesso em: 12 jul. 2020.\\
	
	\noindent \textbf{Como A agricultura será afetada}. [S. l.], 2008. Disponível em: https://www.agritempo.gov.br/climaeagricultura/como-afetara-agricultura.html. Acesso em: 13 jul. 2020.
	
\end{document}