\documentclass[a4paper, 12pt, brazilian]{article}

\usepackage{config}

\begin{document}
	\import{sections/}{titlepage}
	
	\section{Situação 2}
	
	\textit{``Você foi chamado a auxiliar um cafeicultor do norte do Paraná (próximo a Londrina) que ouviu no rádio que uma onda de frio estava chegando à região Sul do país.''}
	
	\begin{enumerate}
		\item\textbf{Onde ele pode conseguir informações sobre essa possível onda de frio? Como ele pode saber se vai ocorrer geada em sua propriedade?}
		
		\hspace{.5cm} A previsão de geadas (com 24 e 48 horas de antecedência e uma tendência para 72 horas) é elaborada pelos meteorologistas do SIMEPAR e apresentada na forma de mapas e textos, os quais descrevem a ocorrência e a intensidade prevista deste fenômeno para todas as regiões do estado do Paraná.
		Essas previsões fazem parte do Alerta Geada, serviço mantido em parceria entre o SIMEPAR e o IAPAR (com a participação da Emater) e estão disponíveis para consulta entre os meses de maio e setembro, que é o período de maior risco de ocorrência deste fenômeno em nosso estado.
		Com as informações de previsão de geadas fornecidas pelo SIMEPAR, o IAPAR e a Emater realizam o aconselhamento técnico para os agricultores visando o melhor enfrentamento deste fenômeno, com o objetivo de mitigar os seus efeitos nas culturas e minimizar as perdas para o agricultor.
		
		UFPR. Simepar. \textbf{Alerta Geada}: Informações. [S. l.], 2020. Disponível em: http://www.simepar.br/prognozweb/simepar/alerta\_geada. Acesso em: 29 jul. 2020.
		
		\item\textbf{Como funciona os alertas e previsões de geadas?}
		
		\item\textbf{Caso se confirme a geada, para as próximas 24 horas, quais as medidas que ele pode tomar para proteger sua plantação?}
		
		Existem algumas estratégias que podem ser adotadas de última hora visando conter os danos ocasionados pela ação das geadas. A primeira delas se baseia na utilização de lonas plásticas ou lençóis de modo que o calor dos raios solares seja contido por mais tempo na vegetação. Todavia, também é necessário realizar o cobrimento do base dos cafezais buscando evitar que o calor armazenado na copa seja perdido. Nessa última é pertinente a utilização de palhada, tendo em vista seu bom isolamento térmico, caso a mesma esteja disponível ou realizar um leve revolvimento do solo para aumentar a proteção das plantas com a prática denominada de chegamento de terra nos troncos.
		
		No estágio de mudas com até 6 meses é possível realizar a dobra das plantas seguido do cobrimento por uma espessa camada de terra, sendo necessário que após ao fim do período de alertas as plantas sejam desenterradas manualmente.
		
		CARAMORI, Paulo; FILHO, Armando; FILHO, Francisco; OLIVEIRA, Dalziza; MORAIS, Reverly; LEAL, Alex; GALDINO, Jonas. \textbf{MÉTODOS DE PROTEÇÃO CONTRA GEADAS EM CAFEZAIS EM FORMAÇÃO}. [S. l.], 07 2020. Disponível em: http://www.iapar.br/arquivos/File/zip\_pdf/protgeada.pdf. Acesso em: 29 jul. 2020.
		
		\item\textbf{Ele gostaria de diversificar sua produção com frutíferas. Quais frutíferas ele poderia plantar em sua propriedade substituindo algumas áreas de café?}
		
		\item\textbf{Sabendo que em cenários futuros de mudanças climáticas há probabilidade de eventos extremos com maior frequência (como geada, secas, e ondas de calor) como esse agricultor poderia se planejar suas atividades para os próximos 15 anos?}
	\end{enumerate}
\end{document}