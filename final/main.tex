\documentclass[a4paper, 12pt, brazilian]{article}

\usepackage{config}

\begin{document}
	\import{sections/}{titlepage}
	
	\section{Situação 2}
	
	\textit{``Você foi chamado a auxiliar um cafeicultor do norte do Paraná (próximo a Londrina) que ouviu no rádio que uma onda de frio estava chegando à região Sul do país.''}
	
	\begin{enumerate}
		\item\textbf{Onde ele pode conseguir informações sobre essa possível onda de frio? Como ele pode saber se vai ocorrer geada em sua propriedade?}
		
		\hspace{.5cm} A previsão de geadas (com 24 e 48 horas de antecedência e uma tendência para 72 horas) é elaborada pelos meteorologistas do SIMEPAR e apresentada na forma de mapas e textos, os quais descrevem a ocorrência e a intensidade prevista deste fenômeno para todas as regiões do estado do Paraná.
		Essas previsões fazem parte do Alerta Geada, serviço mantido em parceria entre o SIMEPAR e o IAPAR (com a participação da Emater) e estão disponíveis para consulta entre os meses de maio e setembro, que é o período de maior risco de ocorrência deste fenômeno no estado.
		Com as informações de previsão de geadas fornecidas pelo SIMEPAR, o IAPAR e a Emater realizam o aconselhamento técnico para os agricultores visando o melhor enfrentamento deste fenômeno, com o objetivo de mitigar os seus efeitos nas culturas e minimizar as perdas para o agricultor.
		
		\textbf{Referência}
		
		UFPR. Simepar. \textbf{Alerta Geada}: Informações. [S. l.], 2020. Disponível em: http://www.simepar.br/prognozweb/simepar/alerta\_geada. Acesso em: 29 jul. 2020.
		
		\item\textbf{Como funciona os alertas e previsões de geadas?}
		
		\hspace{.5cm}No estado do Paraná os alertas se baseiam na compilação das informações de 70 estações meteorológicas tanto da rede do IAPAR, quanto da rede do SIMEPAR. Dessa forma, a partir do monitoramento agroclimático do estado -- uso de mapas -- é feito o balanço das condições apresentadas pelo solo, condições de stress ou déficit hídrico, elementos meteorológicos (precipitação, temperatura) e evapotranspiração, por exemplo. Assim, com base na análise das condições atmosféricas é possível estabelecer o panorama para a ocorrência das geadas com alta taxa de atualização dos dados, e por meio dos veículos de comunicação os agricultores têm acesso aos possíveis riscos que podem correr no período crítico.
		
		\textbf{Referência}
		
		ALERTA geada é ferramenta que ajuda produtores rurais do Paraná. Produção: RIC Rural. YouTube: [s. n.], 2019. Disponível em: https://www.youtube.com/watch?v=A0Il\-WMVrCRA. Acesso em: 5 ago. 2020.
		
		\item\textbf{Caso se confirme a geada, para as próximas 24 horas, quais as medidas que ele pode tomar para proteger sua plantação?}
		
		\hspace{.5cm}Existem algumas estratégias que podem ser adotadas de última hora visando conter os danos ocasionados pela ação das geadas. A primeira delas se baseia na utilização de lonas plásticas ou lençóis de modo que o calor dos raios solares seja contido por mais tempo na vegetação. Todavia, também é necessário realizar o cobrimento da base dos cafezais buscando evitar que o calor armazenado na copa seja perdido. Nessa última é pertinente a utilização de palhada, tendo em vista seu bom isolamento térmico, caso a mesma esteja disponível ou realizar um leve revolvimento do solo para aumentar a proteção das plantas com a prática denominada de chegamento de terra nos troncos.
		No estágio de mudas com até 6 meses é possível realizar a dobra das plantas seguido do cobrimento por uma espessa camada de terra, sendo necessário que após ao fim do período de alertas as plantas sejam desenterradas manualmente.
		\newpage
		\textbf{Referência}
		
		CARAMORI, Paulo; FILHO, Armando; FILHO, Francisco; OLIVEIRA, Dalziza; MORAIS, Reverly; LEAL, Alex; GALDINO, Jonas. \textbf{MÉTODOS DE PROTEÇÃO CONTRA GEADAS EM CAFEZAIS EM FORMAÇÃO}. [S. l.], 07 2020. Disponível em: http://www.iapar.br/arquivos/File/zip\_pdf/protgeada.pdf. Acesso em: 29 jul. 2020.
		
		\item\textbf{Ele gostaria de diversificar sua produção com frutíferas. Quais frutíferas ele poderia plantar em sua propriedade substituindo algumas áreas de café?}
		
		\hspace{.5cm}Tendo em vista o grande potencial das regiões Norte e Noroeste do país para a produção de citros, uma frutífera plausível de ser incorporada à produção seria a laranja. Estudos desenvolvidos pelo Programa Fruticultura do IAPAR têm sido a base para a implantação e condução de pomares de laranja, desta maneira, tendo em vista a característica da região isso colaborou e colabora para a consolidação do estado como grande exportador do produto \textit{in natura} ou na forma de suco.
		
		\hspace{.5cm}Outra cultura frutífera que pode ser manejada é a do pessegueiro, devido ao fato de ser uma espécie que requer um mínimo de horas de frio, o que indica seu plantio em regiões com mais de 600 metros de altitude. Porém, na região Norte é aplicada a quebra de dormência por meios químicos como alternativa às horas de frio.
		
		\textbf{Referência}
		
		IAPAR (PR). \textbf{Programa Fruticultura}: Citricultura. [S. l.], 2020. Disponível em: http://w\-ww.iapar.br/pagina-313.html. Acesso em: 31 jul. 2020.
		
		\textbf{Sistemas de Produção Familiar Praticado no Norte do Paraná: Grãos e Pêssego}. Adenir de Carvalho, Ciro Daniel Marques Marcolini,
		Dimas Soares Júnior, Manuel Pessoa de Lira, Maurílio Soares Gomes, Rafael Fuentes Llanillo, Sergio Luiz Carneiro. Instituto Agronômico do
		Paraná (IAPAR), Instituto Paranaense de Assistência Técnica e Extensão Rural (EMATER). Londrina, 2008. 4 p.
		
		\item\textbf{Sabendo que em cenários futuros de mudanças climáticas há probabilidade de eventos extremos com maior frequência (como geada, secas, e ondas de calor) como esse agricultor poderia se planejar suas atividades para os próximos 15 anos?}
		
		\hspace{.5cm}O planejamento deve ser baseado na análise constante das tendências apresentadas pelos parâmetros de produção da cultura. A partir da parcimônia entre as mudanças ocorrentes nos eventos naturais e as ações tomadas torna-se possível modificar as práticas historicamente utilizadas em períodos de estabilidade e adotar medidas que se enquadrem com a previsão das condições do território para um período mais próximo das análises, tendo em vista a maior certeza dos eventos num curto prazo. 
	\end{enumerate}
\end{document}